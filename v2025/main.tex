
\documentclass[9pt]{developercv} % Default font size, values from 8-12pt are recommended
\usepackage{multicol}
\setlength{\columnsep}{0mm}
%----------------------------------------------------------------------------------------
\usepackage{lipsum}  


\begin{document}

%----------------------------------------------------------------------------------------
%	TITLE AND CONTACT INFORMATION
%----------------------------------------------------------------------------------------

\begin{minipage}[t]{0.5\textwidth} 
	\vspace{-\baselineskip} % Required for vertically aligning minipages
	
	{ \fontsize{16}{20} \textcolor{black}{\textbf{\MakeUppercase{{Холопов Денис}}}}} % First name
	
	\vspace{6pt}
	
	{23 года} % Career or current job title
        {\hfill}\icon{MapMarker}{11}{Санкт-Петербург, Россия}\\
\end{minipage}
\hfill
\begin{minipage}[t]{0.25\textwidth} % 20% of the page width for the first row of icons
	\vspace{-\baselineskip} % Required for vertically aligning minipages
	% The first parameter is the FontAwesome icon name, the second is the box size and the third is the text
	\icon{PaperPlane}{11}{\href{https://t.me/NiceScreen_NiceCode}{@NiceScreen\_NiceCode}}\\ 
    \icon{Phone}{11}{+7 988 532 10 80}\\
    
	
\end{minipage}
\begin{minipage}[t]{0.22\textwidth} % 27% of the page width for the second row of icons
	\vspace{-\baselineskip} % Required for vertically aligning minipages
	
	\icon{Envelope}{11}{\href{mailto:denhol2002@mail.ru}{denhol2002@mail.ru}}\\	
    \icon{Github}{11}{\href{https://github.com/Kre4}{github.com/Kre4}}\\    
    
\end{minipage}


%----------------------------------------------------------------------------------------
%	INTRODUCTION, SKILLS AND TECHNOLOGIES
%----------------------------------------------------------------------------------------

\begin{minipage}[t]{0.46\textwidth}
    \cvsect{Мотивация}
	\vspace{-6pt}
    \begin{minipage}[t]{1\textwidth}
     Хочу развиваться в разработке микросервисов, изучать новые фреймворки, технологии и языки. Улучшить свои навыки в разработке высоконагруженных систем.
     \end{minipage}
\end{minipage}
\hfill % Whitespace between
\begin{minipage}[t]{0.465\textwidth}
    \cvsect{Языки}
    \vspace{-6pt}
    
    % \hspace{26mm} 
    \textbf{Английский} - B2 + чтение документации, 
    
    \textbf{Русский} - родной
    
\end{minipage}


\cvsect{Навыки}
    \vspace{-6pt}

    \begin{minipage}[t]{0.2\textwidth}
        \textbf{Языки:}
    \end{minipage}
    \hfill
    \begin{minipage}[t]{0.73\textwidth}
      Java 8/11/17/21, Kotlin, SQL, Python, TypeScript
    \end{minipage}
    \vspace{2mm}

    \begin{minipage}[t]{0.2\textwidth}
        \textbf{Фреймворки:}
    \end{minipage}
    \hfill
    \begin{minipage}[t]{0.73\textwidth}
      Spring Boot, Spring Cloud Gateway, Spring AMQP, Spring Data JPA, Spring Data JDBC, Spring Boot Actuator, Amazon S3 SDK, Flyway, Liquibase, ModelMapper, Apache POI, MapStruct, Junit5, Testcontainers, Angular
    \end{minipage}
    \vspace{2mm}

    \begin{minipage}[t]{0.2\textwidth}
        \textbf{Безопасность:}
    \end{minipage}
    \hfill
    \begin{minipage}[t]{0.73\textwidth}
      Spring Security, Spring Authorization Server, Keycloak, OAuth2/OIDC
    \end{minipage}
    \vspace{2mm}

    \begin{minipage}[t]{0.2\textwidth}
        \textbf{Хранилища:}
    \end{minipage}
    \hfill
    \begin{minipage}[t]{0.73\textwidth}
      PostgreSQL, MySQL, SQLite, MinIO
    \end{minipage}
    \vspace{2mm}
    
    \begin{minipage}[t]{0.2\textwidth}
        \textbf{Другое:}
    \end{minipage}
    \hfill
    \begin{minipage}[t]{0.735\textwidth}
    Git, OpenAPI, Docker, Apache Kafka, RabbitMQ, S3, Maven, Gradle, Spotless, SpotBugs, SonarCube, Jacoco Report.
    \end{minipage}

%----------------------------------------------------------------------------------------
%	EXPERIENCE
%----------------------------------------------------------------------------------------

\vspace{-0 pt}
\cvsect{Опыт работы}
\begin{entrylist}
    \entry
		{11/2024 - сейчас}
        {Middle Backend Developer}
        {КРИТ}
      {\vspace{-10pt}
        \begin{itemize}[noitemsep,topsep=0pt,parsep=0pt,partopsep=0pt, leftmargin=-1pt]
            \item {Разработка и поддержка B2B приложения для технического обслуживания и ремонта оборудования, работающего на микросервисной архитектуре.}
            \item {Проектирование и полная разработка 4 микросервисов с нуля (Java/Kotlin, Spring Boot), каждый с покрытием тестами от 70\%, полной документацией OpenAPI и настройками code quality (Spotless, SpotBugs).}
            \item {Разработал микросервис управления файлами на базе S3-совместимого хранилища, который стал переиспользуемым компонентом и интегрирован в 3+ существующих продукта компании.}
            \item {Автоматизировал критичные бизнес-процессы:}
                \begin{itemize}[noitemsep,topsep=0pt,parsep=0pt,partopsep=0pt, leftmargin=3pt]
                    \item Заменил ручную синхронизацию между тремя системами на асинхронную интеграцию через message broker (RabbitMQ), сократив время синхронизации и ошибки. 
                    \item Реализовал автоматическое извлечение API endpoints из OpenAPI спецификаций для динамического управления ролевой моделью доступа.
                \end{itemize}
            \item {Разработка и настройка API Gateway сервисов на Spring Cloud Gateway для нескольких команд. Функционал по маршрутизации, аутентификации и rate limiting.}
            \item {Активное участие в архитектурных обсуждениях и принятии технических решений, code review в команде.}
        \end{itemize} 
        % \texttt{} \slashsep \texttt{} \slashsep \texttt{} TODO ключевые слова?
        }
    \entry
		{12/2021 - 11/2024}
        {Junior Fullstack Developer}
        {ВНИИГ им. Б.Е. Веденеева}
      {\vspace{-10pt}
        \begin{itemize}[noitemsep,topsep=0pt,parsep=0pt,partopsep=0pt, leftmargin=-1pt]
            \item {Разработка группы проектов для мониторинга состояния гидроэлектростанций в реальном времени}
            \item {Backend: Java 8/11, Spring Framework (Boot, MVC, Data JPA), PostgreSQL, Apache Kafka. Frontend: TypeScript, Angular}
            \item {Проектирование и оптимизация архитектуры БД}
            \item {Проектирование и разработка сервисов для интеграции с измерительной аппаратурой. Решение задач по получению и передаче данных, их обработке}
            \item {Оптимизация производительности: ускорил старт приложения с 5 до 3 минут путем оптимизации инициализации и lazy loading}
            \item {Code review и менторство стажеров}
        \end{itemize} 
        % \texttt{Spring} \slashsep \texttt{Apache Kafka} \slashsep \texttt{PostgreSQL} TODO ключевые слова?
        }
\end{entrylist}

%----------------------------------------------------------------------------------------
%	EDUCATION
%----------------------------------------------------------------------------------------
\vspace{-10 pt}
\cvsect{Образование}
\begin{entrylist}
    \entry
		{2025 - 2027гг.}
		{НИУ ИТМО, магистратура}
		{}
		{ФИТиП, Программирование и интернет-технологии}
    \entry
		{2020 - 2024гг.}
		{НИУ ИТМО, бакалавриат}
        {}
		{ФИТиП, Информационные системы и технологии}
    

    
\end{entrylist}

\vspace{-20 pt}
\cvsect{\textbf{Олимпиады}}
\begin{entrylist}
    \entry
        {2023/24г.}
        {Промышленный бэкенд, студенческая олимпиада}
        {призёр}
        {\href{https://drive.google.com/file/d/1YQdVSubT0tgrUXBRzmaYQZjn4SfmLXu-/view}{Мегаолимпиада ИТМО}}
    \entry
        {2023/24г.}
        {Мобильная разработка, студенческая олимпиада}
        {призёр}
        {\href{https://drive.google.com/file/d/1yrZl55YZbbO_mYd815QeP8cv3__eWc_P/view}{Мегаолимпиада ИТМО}}
    \entry
		{2019/20г.}
        {Информатика, 11 класс.}
        {победитель}
        {\href{https://diploma.rsr-olymp.ru/files/rsosh-diplomas-static/compiled-storage-2020/by-code/184635984832/color.pdf}{Открытая олимпиада школьников}}
    \entry
		{2019/20г.}
        {Математика, 11 класс.}
        {призёр III степени}
		{\href{https://diploma.rsr-olymp.ru/files/rsosh-diplomas-static/compiled-storage-2020/by-code/186308281722/color.pdf}{Отраслевая физико-математическая олимпиада школьников «Росатом»}}
\end{entrylist}




%----------------------------------------------------------------------------------------
%	Projects
%----------------------------------------------------------------------------------------
\vspace{-10 pt}

\cvsect{Проекты}
\begin{entrylist}
     \entry
		{Jpa/PostgreSQL}
		{BatchInsert - Анализ производительности batch-операций}
		{\href{https://github.com/Kre4/BatchInsert}{github}}
		{\vspace{-10pt}
        \begin{itemize}[noitemsep,topsep=0pt,parsep=0pt,partopsep=0pt, leftmargin=-1pt]
            \item Исследование производительности batch-вставок в Hibernate + PostgreSQL
            \item Выявление и решение проблем с конфигурацией батчинга, влияющих на производительность продакшн-приложения
        \end{itemize}}

    \entry
		{Java/Spring Boot}
		{itmo-backend-olymp - Промышленный бэкенд}
		{\href{https://github.com/Kre4/itmo-backend-olymp}{github}}
		{\vspace{-10pt}
        \begin{itemize}[noitemsep,topsep=0pt,parsep=0pt,partopsep=0pt, leftmargin=-1pt]
            \item Решение задач промышленного бэкенда для олимпиады ИТМО
            \item RESTful API сервис, позволяющий управлять списком кинофильмов и режиссеров
        \end{itemize}}

    \entry
		{Kotlin/JDBC}
		{repository-dsl}
		{\href{https://github.com/Kre4/repository-dsl}{github}}
		{\vspace{-10pt}
        \begin{itemize}[noitemsep,topsep=0pt,parsep=0pt,partopsep=0pt, leftmargin=-1pt]
            \item Экспериментальная библиотека с Kotlin DSL для улучшения читаемости кода работы с репозиториями
            \item Демонстрация применения функциональных возможностей Kotlin для создания выразительного API
        \end{itemize}}
        
    \entry
		{Kotlin/Plugin}
		{CodeCleaner}
		{\href{https://github.com/Kre4/CodeCleaner}{github}}
		{\vspace{-10pt}
        \begin{itemize}[noitemsep,topsep=0pt,parsep=0pt,partopsep=0pt, leftmargin=-1pt]
            \item Плагин для IDEA, реализующий радикальную чистку кода без гарантии последующей компиляции
        \end{itemize}}
\end{entrylist}


\vspace{-10 pt}
\cvsect{Курсы}
\begin{entrylist}
	\entry
        {Базовые вузовские}
		{}
		{}
        {Алгоритмы и структуры данных, Web-программирование Машинное обучение, Основы тестирования программного обеспечения.
        Математический анализ, Линейная алгебра, Дискретная математика, Теория вероятностей и математическая статистика.}
	\entry
		{Углублённые вузовские}
        {}
		{}
		{Объектно-ориентированное программирование, Базы данных, Архитектура информационных систем,
Анализ и проектирование на UML, Проектирование программного обеспечения.}
	

\end{entrylist}

\end{document}
